\documentclass[12pt, a4paper]{article}
\usepackage[polish]{babel}
\usepackage[T1]{fontenc}
\usepackage[utf8]{inputenc}
\usepackage{mathtools}
\usepackage{amsfonts,amsmath,amssymb,amsthm}
\usepackage{enumerate}
\usepackage{hyperref}

\title{Optymalizator}
\author{Stanisław Bitner, Mikołaj Bulge, Jakub Dawidowicz,\\Antoni Grodowski, Szymon Mrozicki}
\date{\today}

\begin{document}
\maketitle

\section{Motywacja}
Jednym z kluczowych problemów polskiej farmaceutyki jest brak
prostego i jednolitego systemu służącego do obsługi oficjalnych
danych dostarczanych przez Ministerstwo Zdrowia i Narodowy Fundusz
Zdrowia. Dostęp do wszelkich informacji dotyczących leków i ich
refundacji wiąże się, z męczącym i niezwykle czasochłonnym procesem
znalezienia ich wewnątrz chaotycznej i nieuporządkowanej bazy danych.
Co więcej, znalezione informacje bywają nieaktualne lub ze sobą
sprzeczne. W obliczu przeciążonego systemu opieki zdrowotnej w Polsce
personel medyczny nie może sobie pozwolić na tak nieefektywne metody
wyszukiwania danych. Obecnie nie ma na rynku prostego w obsłudze
oprogramowania wychodzącego naprzeciw powyższym problemom.

\section{Opis}
\textit{\href{https://polmed.pl}{PolMed}} jest firmą farmaceutyczną,
która od wielu lat doskonali różne metody na wytwarzanie
i dystrybucję leków. Brakuje jej jednak nowoczesnego narzędzia
umożliwiającego prostą i efektywną komunikację między poszczególnymi
sektorami. Nasz zespół poświęcił wiele czasu na analizę obecnej
sytuacji na rynku oraz wymagań personelu medycznego i pacjentów.
Dzięki temu jesteśmy w stanie zaproponować wstępny szkic aplikacji,
która w nowoczesny i rewolucyjny sposób rozwiąże powyższe problemy.
\begin{enumerate}
    \item Poprzez natłok koncernów farmaceutycznych na rynku w ostatnich latach
      pojawiło się niezliczenie wiele nowych leków. Siłą rzeczy, mając na
      uwadze ograniczoną liczbę chorób, leki te często są sobie równoważne ze
      względu na substancję czynną, ale niekoniecznie w kwestii ceny. Z tego
      powodu pacjenci stoją przed skomplikowanym problemem optymalizacji
      kosztów leczenia. Proces ten zostanie ułatwiony dzięki naszej aplikacji,
      która po wprowadzeniu danych będzie dokonywać wielokryterialnej analizy
      potrzeb pacjenta i według tego proponować zestaw najbardziej opłacalnych
      planów kuracji.
    \item Niestety ze względu na chaotyczne prowadzenie dokumentacji leków
      przez podmioty zajmujące się sektorem zdrowotnym, niekiedy bardzo trudno
      jest znaleźć szukane informacje o interesujących nas preparatach.
      Proponujemy stworzenie wewnętrznego systemu informacji o lekach, który
      zawierałby zintegrowane informacje z wielu źródeł danych. Dzięki temu
      informacje o wyszukiwanych preparatach byłyby pełne i spójne.
    \item Leki często zawierają więcej niż jedną substancję czynną. Sprawia to,
      że dany preparat może działać pozytywnie na jedną chorobę, ale powodować
      niepożądane skutki w połączeniu z inną dolegliwością pacjenta. Sprawia
      to, że lekarz podczas wypisywania recepty musi łączyć wiele informacji na
      temat przeciwwskazań, co nie tylko zabiera cenny czas, ale może także
      doprowadzić do niepożądanych efektów. Dlatego proponujemy w aplikacji
      funkcjonalność, która według wprowadzonych przez użytkownika chorób
      pacjenta, będzie proponować leki, które nie będą konfliktować
      z dolegliwościami leczonego.
    \item Gdy kuracja pacjenta wymaga większej ilości leków, często można
      się spotkać, z tym że w danej aptece nie można skompletować wymaganej
      listy preparatów. Wówczas kupujący musi poświęcić wiele czasu na
      sprawdzenie dostępności w wielu aptekach i zaplanowaniu odpowiedniej
      kolejności odwiedzenia placówek farmaceutycznych. Problem ten rozwiążemy
      poprzez automatyczne wielokryterialne (np. ze względu na koszt lub czas)
      wyznaczanie kilku odpowiednich tras.
    \item Epidemia COVID uświadomiła nam, jak wielu ludzi preferuje zakupy przez
      Internet. W szczególności w przypadku osób chorych jest to niezwykle
      istotne, ponieważ często potrzebują one danego zbioru leków, ale nie mają
      możliwości wyjścia z domu. Proponowana przez nas aplikacja
      przeszukiwałaby apteki internetowe i automatycznie dokonywałaby
      zamówienia optymalnego ze względu na koszt lub czas dostawy produktów
      dodanych z ogólnej puli leków dodanych do koszyka.
\end{enumerate}

Proponowany projekt będzie korzystać między innymi z danych udostępnionych
przez Ministerstwo Zdrowia na temat
\href{https://www.gov.pl/web/zdrowie/obwieszczenia-ministra-zdrowia-lista-lekow-refundowanych?fbclid=IwAR3kUJzPeXQa_i6aKzo8EIS-HuLgArTdgc19OfAbQsrTF7Xd2xNiUIUym7c}{leków
refundowanych}. Ponadto kwestie prawne związane z branżą farmaceutyczną są
wyjątkowo skomplikowane (w szczególności te dotyczące refundacji oraz
zamienników leków), więc podczas rozwijania projektu będziemy korzystać
z \href{https://isap.sejm.gov.pl/isap.nsf/download.xsp/WDU20111220696/U/D20110696Lj.pdf}{podstaw
prawnych} opublikowanych przez sejm. 

\section{Podsumowanie}
Stopień skomplikowania projektu jest na tyle wysoki, że jego realizację będzie
przebiegała etapowo. Po przeprowadzeniu dogłębnej analizy potrzeb firmy PolMed
nasi eksperci zdecydowali, że najwyższym priorytetem są problemy, z którymi
zmagają się pacjenci. W wyniku kryzysu gospodarczego nadzwyczaj ważne stało się
oszczędzanie. W związku z tym pierwszą fazą projektu będzie dostarczenie
narzędzia do optymalizacji kosztów leczenia.

\end{document}
