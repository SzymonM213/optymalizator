\documentclass[12pt]{article}
\usepackage{graphicx} % Required for inserting images
\usepackage[T1]{fontenc}
\usepackage[polish]{babel}
\usepackage[utf8]{inputenc}
\usepackage{hyperref}

\title{Optymalizator -- opis wymagań}
\author{Stanisław Bitner, Mikołaj Bulge, Jakub Dawidowicz,\\Antoni Grodowski, Szymon Mrozicki}
\date{\today}

\begin{document}

\maketitle

\section{Wstęp}

Użytkownik chcący poznać informacje na temat zamienników wybranego leku
refundowanego poprzez wyszukiwarkę aplikacji WWW otrzymuje pełną listę
dostępnych odpowiedników. Przedstawione dane mogą być  sortowane według potrzeb
użytkownika.

\section{Wymagania funkcjonalne}

\subsection{Wyszukiwarka leku refundowanego.}

\begin{enumerate}
    \item Użytkownik, posiadając urządzenie z dostępem do internetu oraz
      przeglądarką Chrome 109 lub Firefox 102.9.0 bądź nowszą ma możliwość
      korzystania z aplikacji.
    \item Użytkownik korzysta z wyszukiwarki, wpisując w pole wyszukiwania
      odpowiednie frazy opisane w poniższym podpunkcie.
    \item Użytkownik może wyszukać lek refundowany podając kod EAN leku. Innym
      sposobem wyszukania jest podanie takich parametrów jak: nazwa, zawartość
      opakowania, substancja czynna. Dokładniej -- po wpisaniu słów w pole
      wyszukiwania, każdy wynik musi zawierać wszystkie te słowa. Wielkości
      liter oraz znaki polskie nie mają znaczenia. Z dostępnej listy użytkownik
      wybiera interesujący go lek. W przypadku, gdy lista wynikowa jest pusta,
      użytkownik otrzyma komunikat o braku dostępnych zamienników.
    \item Kolejność proponowanych leków jest zależna od częstotliwości, z jaką
      były one wyszukiwane przez wcześniej korzystających z aplikacji
      użytkowników. W przypadku, gdy leki są wyszukiwane z równą
      częstotliwością, o kolejności proponowanych leków decyduje kolejność
      alfabetyczna.
    
\end{enumerate}

\subsection{Prezentacja listy zamienników.}

\begin{enumerate}
    \item Po skorzystaniu z wyszukiwarki i znalezieniu odpowiedniego leku
      użytkownik otrzymuje listę leków refundowanych, które mogą służyć jako
      zamienniki.
    \item Znalezione zamienniki wyświetlane są w kolejności od najtańszych do
      najdroższych. Użytkownik według preferencji może wybrać, czy dane mają
      być posortowane alfabetycznie, według ceny, czy wielkości opakowania,
      jeżeli opakowanie jest przeznaczone na konkretną liczbę aplikacji leku.
    \item Leki, na temat których nie ma wystarczająco precyzyjnych informacji
      w bazie, nie będą obsługiwane przez aplikację. Aplikacja obsługuje co
      najmniej 95\% wszystkich leków z bazy danych, a o nieobsługiwaniu
      wybranego leku informuje użytkownika.
    \item Jeżeli pewien lek nie jest obsługiwany, to wszystkie leki zawierające
      tę samą substancję czynną również nie są obsługiwane przez aplikację.
    \item W przypadku, gdy któryś z wyszukanych przez aplikację leków
      zamiennych, z powodu nieprecyzyjności informacji w bazie, nie może
      jednoznacznie zostać zakwalifikowany do grupy zamienników, lub odrzucony,
      użytkownik zostanie poinformowany o możliwości istnienia zamiennika.
\end{enumerate}

\subsection{Lista wspieranych jednostek i ich skrótów}

Użytkownicy na stronie aplikacji będą mieć możliwość otworzenia pliku
tekstowego z wymienionymi jednostkami (oraz ich skrótami), obsługiwanymi przez
wyszukiwarkę aplikacji, służącymi do wyrażania policzalności leku.

\section{Wymagania niefunkcjonalne}

\subsection{Połączenie z serwerem.}

Użytkownik połączony z internetem o przepustowości co najmniej 4 mb/s może korzystać z aplikacji połączonej z serwerem zewnętrznym.

\subsection{Baza danych.}

 Aplikacja w celu przechowywania danych będzie korzystać z bazy danych PostgreSQL.
 Wyniki wyszukiwania aplikacji będą przedstawione w takiej samej formie, jak
 w oryginalnym zestawieniu leków refundowanych, które można znaleźć na krajowej
 \href{https://www.gov.pl/attachment/e1c2cf1e-8fed-468a-9f1f-e6d49f5b137f}{liście
 leków (www.gov.pl)}. Dane, z których będzie korzystała aplikacja, będą
 pochodzić z marca 2023 i nie będą aktualizowane na bieżąco.

\subsection{Szybkość wskazywania wyników wyszukiwania.}
 
Zapytania o wyniki wyszukiwań będą obsługiwane w mniej niż jedną sekundę.

\subsection{Przepustowość aplikacji.}

Z aplikacji może korzystać co najwyżej 100 użytkowników jednocześnie, w przeciwnym wypadku czas wyszukiwań lub niezawodność aplikacji nie muszą zostać spełnione.

\section{Definicja odpowiednika leku}

% Lek jest odpowiednikiem drugiego leku wtedy i tylko wtedy, gdy
% Niech $p, q$ będą lekami.
% Mówimy, że
% $$p \in odp(q) \iff subst(p) = subst(q) \land dawka(p) \in [0.9 \cdot dawka(q), 1.1 \cdot dawka(q)$$
Odpowiednik to lek zawierający taką samą substancję czynną w takiej samej dawce
i takiej samej postaci jak lek oryginalny. Rozmiar opakowania odpowiednika,
czyli łączna zawartość substancji czynnej w opakowaniu nie różni się o więcej
niż 10\% względem zawartości oryginalnego leku. Wskazania refundacyjne są takie
same.

\end{document}
